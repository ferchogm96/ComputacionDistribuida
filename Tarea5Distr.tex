\documentclass[letterpaper,10pt]{article}

% Soporte para los acentos.
\usepackage[utf8]{inputenc}
\usepackage[T1]{fontenc}
% Idioma español.
\usepackage[spanish,mexico, es-tabla]{babel}
% Soporte de símbolos adicionales (matemáticas)
\usepackage{multirow}
\usepackage{amsmath}
\usepackage{amssymb}
\usepackage{amsthm}
\usepackage{amsfonts}
\usepackage{latexsym}
\usepackage{enumerate}
\usepackage{ragged2e}
% Soporte para imágenes.
\usepackage{graphicx}
% Soporte para código.
\usepackage{listings}
% Soporte para URL.
\usepackage{hyperref}
\usepackage[all]{xy} %para diagramas conmutativos
% Modificamos los márgenes del documento.
\usepackage[lmargin=2cm,rmargin=2cm,top=2cm,bottom=2cm]{geometry}
\title{Computación Distribuida \\ Tarea 5}
\author{González Montiel Luis Fernando \\}
\date{\today}

\begin{document}
\maketitle

	\begin{enumerate}

	 % Ejercicio 1.
    \item Sean A y B dos procesos cuyos relojes no estan sincronizados pero ambos tienen drift acotado.
     Cuando A manda un mensaje a B, el tiempo maximo real que tarda en llegar el mensaje es a lo mas D.
     Supon que tienes un algoritmo en el que A manda mensajes cada T unidades de su tiempo. 
     Para ir midiendo las unidades de tiempo, A tiene un contador que aumenta cada tanto 
     (evento de computo local).\\
     \\
     Dibuja una ejecución $\alpha$ donde A manda 3 mensajes a B.\\
     \textsc{Solución:}\\
     \begin{center}
     \includegraphics[width=0.7\textwidth]{ejecucion1.png}\\
     \end{center}
	 
	 De la ejecución $\alpha$, menciona cuáles son los eventos y su relación de causalidad:\\
	 \textsc{Solución:}
     \\
     Eventos: $\phi$1, $\phi$2, $\phi$3, $\phi$4, $\phi$5, $\phi$6\\
	Relación de Causalidad:\\
	$\phi$1 => $\phi$2\\
	$\phi$2 => $\phi$3\\
	$\phi$4 => $\phi$5\\
	$\phi$5 => $\phi$6\\
	\\
	$\phi$1 => $\phi$4\\
	$\phi$2 => $\phi$5\\
	$\phi$3 => $\phi$6\\
	\\
	$\phi$1 => $\phi$3\\
	$\phi$4 => $\phi$6\\
	\\
	¿Cuánto tiempo puede tardar a lo mas en terminar $\alpha$?

	 \textsc{Solución:}\\
	 $\phi$1 => $\phi$2 + $\phi$2 => $\phi$3 + $\phi$4 => $\phi$5 + $\phi$5 => $\phi$6 \\
 
    % Ejercicio 2.
   \item Considera la ejecucion $\alpha$ de la Figura 2. Los retardos maximos para cada uno de los eventos son\\ 
   los siguientes: a $\longrightarrow$ b es Ba,b, b $\longrightarrow$ c es Bb,c, c $\longrightarrow$ d es Bc,d, d $\longrightarrow$ e es Bd,e, a $\longrightarrow$ e es Ba,e y b $\longrightarrow$ d es\\	   Bb,d. Los retardos mínimos son similares, pero invirtiendo el orden de las letras en B, i.e. el mínimo\\
	tiempo que puede tomar en pasar a $\longrightarrow$ b es Bb,a.
	\\	
	¿Cuánto vale el retardo maximo de c a e en términos de las B0 s? Es decir, dado z(e) - z(c) $\leq$ x ,\\
	¿Cuánto vale x?\\
    \textsc{Solución:}
    \\
	Bc,d + Bd,e
	\\
	\\
	¿Cuánto vale el retardo mínimo de toda la ejecución?\\
	\textsc{Solución:}
    \\
	Bb,a + Bc,b + Bd,c + Be,d \\
	\\
	Anotar los eventos con los relojes lógicos y los relojes vectoriales.\\
	\textsc{Solución:}
    \\
	\begin{center}
    \includegraphics[width=0.7\textwidth]{reloj1.png}\\
    \end{center}
    
    \begin{center}
    \includegraphics[width=0.7\textwidth]{reloj2.png}\\
    \end{center}
	
	% Ejercicio 3.
   \item Sea $\alpha$ una ejecución y sea $\phi$1 y $\phi$2 dos eventos en $\alpha$. Si $\phi$1 => $\phi$2 entonces LT($\phi$1) < LT($\phi$2)
	
    
	
	\end{enumerate} 
\end{document}
