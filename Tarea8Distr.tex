\documentclass[letterpaper,10pt]{article}

% Soporte para los acentos.
\usepackage[utf8]{inputenc}
\usepackage[T1]{fontenc}
% Idioma español.
\usepackage[spanish,mexico, es-tabla]{babel}
% Soporte de símbolos adicionales (matemáticas)
\usepackage{multirow}
\usepackage{amsmath}
\usepackage{amssymb}
\usepackage{amsthm}
\usepackage{amsfonts}
\usepackage{latexsym}
\usepackage{enumerate}
\usepackage{ragged2e}
% Soporte para imágenes.
\usepackage{graphicx}
% Soporte para código.
\usepackage{listings}
% Soporte para URL.
\usepackage{hyperref}
\usepackage[all]{xy} %para diagramas conmutativos
% Modificamos los márgenes del documento.
\usepackage[lmargin=2cm,rmargin=2cm,top=2cm,bottom=2cm]{geometry}
\title{Computación Distribuida \\ Tarea 8}
\author{González Montiel Luis Fernando \\}
\date{\today}

\begin{document}
\maketitle

	\begin{enumerate}

	 % Ejercicio 1.
    \item Recuerda la reducción dada en clase para transformar cualquier detector de fallas que cumpla integridad débil en uno que cumple integridad fuerte.\\
    Algorítmo1
    \begin{lstlisting}
output p = null
while true do
p queries its local failure detector module D p
suspects p = D p
send(p, suspects p ) to all
end while
upon receiving (q, suspects q ) from neighbor q
begin:
output p = (output p u suspects q ) - {q}
end
	\end{lstlisting}
   
Demostrar que el algoritmo de transformación es correcto, es decir, demostrar que el detector de fallas simulado satisface integridad fuerte y que mantiene las propiedades de precisión (perpetua o eventual).\\
   
    \textsc{Solución:}
    \\  
    Si algún proceso p falla, hay un correcto proceso q que eventualmente sospecha de p (integridad débil). Cuando q envía su lista de sospechosos a cada proceso, p se marca como sospechoso y cada proceso agrega p a su lista de sospechosos. Como p falló, ahí será el momento en el que cada proceso deja de recibir mensajes de p, por lo que no eliminan p de su lista de sospechosos nuevamente.

Para probar que se preserva la precisión perpetua, ya que ningún proceso sospecha p antes del tiempo, ningún proceso envía un mensaje con p en la lista de sospechosos antes de que p se crashe (si lo hace). Para una precisión eventual, si p es correcto y todos los procesos correctos no sospechan p, y hay algún proceso que sospecha p, entonces eventualmente falla. En otro caso, si p es correcto, eventualmente cada proceso correcto q recibe nuevamente un mensaje de p, luego q elimina p de la salida q. Dado que no hay procesos correctos sospecha p, ningún proceso envía p a la lista de sospechosos. Entonces, q no agrega p a la salida q nuevamente.

    % Ejercicio 2.
    \item Recuerda la implementación de $\diamond$P vista en clase (Algoritmo 1).\\
Suponemos que el sistema es parcialmente síncrono, es decir, eventualmente se estabilizan las velocidades relativas de los procesos y las velocidades de transmisión de los canales; y que todos los procesos pueden comunicarse entre ellos.
Demuestra que la implementación vista cumple con integridad fuerte.
\\
    
   
    \textsc{Solución:}
    \\	
	Bajo la premisa  que siempre hay un proceso correcto que manda la información correcta y aunque fallen n procesos, siempre habrá uno que siga mandando la información correcta por los Lemas ya vistos.
	Entonces poco a poco se irá creando un conjunto de procesos correctos que tengan información correcta.\\
    
    % Ejercicio 3.
    \item Si tuviéramos un sistema con topología arbitraria, ¿el algoritmo del inciso pasado funcionaría? Argumenta tu respuesta.\\
    \begin{lstlisting}
    Constants
neighbors
T
n
Variables
clock()
lastHB = [0, . . . , 0]
process
timeout = [T, . . . , T ]
suspect = [false, . . . , false]

every T units of time
for each q \in neighbors do
   send(hHB, pi)
end for

upon the receiving of hHB, qi from a neighbor q
begin:
   if (suspect[q] = true) then
      timeout[q] = timeout[q] + 1
      suspect[q] = f alse
   end if
   lastHB[q] = clock()
end

when timeout[q] == clock() - lastHB[q] . when the timer for a message of
q \in \prod expires, p starts suspecting q
begin:
   suspect[q] = true
end
    \end{lstlisting}
    
    \textsc{Solución:}
    \\
    Suponiendo que se habla de cualquier gráfica, es una disconexa no se podría porque no llegan los mensajes.
    
	 
	\end{enumerate} 
\end{document}
